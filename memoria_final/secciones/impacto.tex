\chapter{Análisis de impacto}

A continuación se descibirán los impactos de los objetivos que se han logrado con este proyecto, relacionándolos cuando sea posible con los Objetivos de Desarrollo Sostenible de la ONU.





En la memoria del TFG se ha incluye un capítulo en el que el/la estudiante debe analizar el impacto de su trabajo, vinculándolo cuando sea posible con los Objetivos de Desarrollo Sostenible



En este capítulo se realizará un análisis del impacto potencial de los resultados
obtenidos durante la realización del TFG de cara a los objetivos marcados por la
Agenda de 2030 según la ODS.
Para hablar de la Agenda de 2030, primero debemos saber qué es, por lo que
haré un breve resumen.
La ONU (Organización de las Naciones Unidas) adoptó en Septiembre de 2015
un conjunto de objetivos para erradicar la pobreza, luchar contra el cambio
climático y así proteger el planeta entre otras cosas.
Para que esto se pueda conseguir será necesario que ciudadanos, gobiernos y
empresas cumplan con su parte de aquí a 15 años, por eso el nombre Agenda
2030. Los objetivos se han organizado en una lista de 17, que a su vez se dividen
en diferentes tareas.
De cara a la Agenda 2030, el presente TFG haría especial hincapié en el punto
nº4 y el nº8, que hacen referencia a la educación de calidad, al trabajo decente
y al crecimiento económico. Concretamente se desarrolla una herramienta para
que todos aquellos que lo necesiten puedan tener acceso a una educación libre
y gratuita, desde cualquier parte del mundo; esto deriva en un desarrollo del
conocimiento que hará que, más adelante, se integren en el mercado laboral.
Se apuesta por una educación más completa, que es la base para el desarrollo
de cualquiera de los puntos que se establecen en el tratado. Por ejemplo, in-
tegrar el cuidado del clima, la vida submarina o los ecosistemas terrestres es
fácil si previamente se ha apostado por unas comunidades sostenibles y una
industria innovadora, lo cual solo se consigue manejando mucha información,
teniendo acceso a la documentación y favoreciendo la transmisión de valores y
conocimientos gracias al sistema desarrollado en este TFG.

