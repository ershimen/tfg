\chapter{Análisis de impacto}

En este capítulo se anañizará primero el impacto de los objetivos que se han logrado con este proyecto, clasificándolos según el ámbito al que afectan, y posteriormente se relacionarán con los Objetivos de Desarrollo Sostenible de la ONU.

\section{Impacto personal}

A nivel personal el impacto de este Trabajo de Fin de Grado se ve reflejado sobre todo en la cantidad de conocimientos adquiridos durante la realización del mismo. Estos conocimientos no se basan únicamente en el uso de tecnologías nuevas, si no también en la experiencia que ha supuesto implementar todos los componentes.

\section{Impacto empresarial y económico}

La implementación propuesta es de utilidad para los trabajadores de las empresas que implementan sistemas distribuidos, puesto que facilita el proceso de depuración. Esto implica que los trabajadores pueden completar las tareas de este tipo con mayor velocidad, lo cual tendría una repercusión positiva sobre la empresa.

\section{Impacto cultural}
Además de facilitar la labor de los programadores de aplicaciones distribuidas, este proyecto también mejora la calidad de la docencia en las asignaturas de esta materia. Disponer de un recurso que facilite la explicación de los algoritmos de la materia supone que los estudiantes tendrán más facilidades para comprenderlos.

\section{Relación con los Objetivos de Desarrollo Sostenible}

En 2015 el Ministerio de Derechos Sociales publicó la \textit{Agenda 2030}\cite{agenda2030}, basada en los Objetivos de Desarrollo Sostenible\cite{ods} de la ONU, cuya finalidad es mejorar las condiciones de vida de los ciudadanos. Su nombre se deriva de que en los próximos 15 años (a partir de 2015) el Gobierno debe adoptar medidas para luchar contra el cambio climático y erradicar la pobreza, entre otros. La \textit{Agenda 2030} recoge en 17 apartados los diferentes Objetivos de Desarrollo Sostenible. 

Analizando los objetivos del proyecto, se puede deducir que están estrechamente relacionados con el punto ''Educación de calidad'' de la \textit{Agenda 2030}. Uno de los objetivos principales de este Trabajo de Fin de Grado es proporcionar una herramienta que facilite la docencia en la asignatura de Sistemas Distribuidos y facilite la depuración el ámbito de la programación distribuida. Esto por una parte implicaría una docencia de mayor calidad en las asignaturas de esta materia, y por otra parte tabién mejoraría el desarrollo de aplicaciones distribuidas. Este último aspecto también está relacionado con el punto ''Trabajo Decente y Crecimiento Económico'' de los objetivos de Desarrollo Sostenible, dado que la solución propuesta en este proyecto es una herramienta de apoyo para los programadores de aplicaciones distribuidas, lo cual contribuye a un trabajo de mejor calidad.
