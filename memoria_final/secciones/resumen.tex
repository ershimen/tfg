\chapter*{Resumen}

Cuando se habla de sistemas distribuidos, una de las propiedades fundamentales de los algoritmos de esta disciplina es que se ejecutan en múltiples máquinas (nodos) concurrentemente. Estos nodos interactúan de alguna manera entre ellos para lograr un objetivo determianado. Por lo general, la simultaneidad de los eventos causa dificultades a los estudiantes de la materia a la hora de entender el funcionamiento de estos algoritmos puesto que es complicado mantener una visión global del estado del sistema.

Este proyecto tiene como objetivo principal proporcionar una herramienta que facilite la comprensión del funcionamiento de los algoritmos de la asignatura de Sistemas Distribuidos. A diferencia de otras implementaciones que simplemente visualizan una simulación el algoritmo, en esta se pretende visualizar una ejecución real. De esta forma se podrá también ofrecer la posibilidad de interactuar con los nodos en tiempo de ejecución, pudiendo observar cómo el algoritmo reacciona ante distintas situaciones. Además de ofrecer una solución que facilita el aprendizaje en la asignatura de Sistemas Distribuidos, también se podría emplear como una herramienta de depuración de estos algoritmos.

\newpage

\chapter*{Abstract}

When it comes to distributed systems, one of the most important aspects of the algorithms is that they are executed simultaneously in various independent machines, commonly referred as nodes. These nodes interact with each other to achieve some objective. The simultaneousness of the events usually causes difficulties to the students when they try to understand the behavior of the algorithm. This is due to the fact that it is hard to maintain a global vision of the state of the distributed system.

The main goal of this project is to provide a tool that facilitates the understanding of the algorithms of the distributed system discipline. Contrary to other existing implementations that only simulate a visualization of the algorithm, this one will visualize a real execution, permitting real-time interaction with the nodes. This allows to visualize how the algorithm reacts in different situations. Additionally, this tool provides a debugging environment for this type of algorithms.


%%%%%%%%%%%%%%%%%%%%%%%%%%%%%%%%%%%%%%%%%%%%%%%%%%%%%%%%%%%
%% Final del resumen. 
%%%%%%%%%%%%%%%%%%%%%%%%%%%%%%%%%%%%%%%%%%%%%%%%%%%%%%%%%%%