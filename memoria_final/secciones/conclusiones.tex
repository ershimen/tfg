\chapter{Conclusiones y trabajo futuro}

En este último capítulo se describen las conclusiones personales con respecto a la realización de este proyecto. Recapitulando los objetivos del proyecto indicados en la introducción, se pretende implementar una aplicación que permita visualizar una ejecución real de un algoritmo distribuido, permitiendo interactuar con los nodos. Además de ofrecer la posibilidad de intercambiar fácilmente el algoritmo a visualizar. Como se ha justificado en los apartados del desarrollo, el diseño y la implementación cumple con estos tres requisitos del proyecto.

Dejando a un lado los objetivos mencionados en la introducción, este proyecto también ha supuesto un reto en otros aspectos. Para empezar, todos los lenguajes de implementación excepto Python eran desconocidos. Aprender HTML, CSS, JavaScript y Go en unos 4 meses ha sido bastante desafiante. Además de aprender sobre lenguajes concretos, también se han empleado herramientas de estos, por ejemplo Electron CITAR, que también requieren un periodo de formación.



Para la realización de estas tareas ha sido necesario el estudio y aprendizaje de
tecnologías como HTML, Elixir y Phoenix, las cuales hasta el momento de iniciar
37
7.2. Resultados
el trabajo desconocía, sin embargo ha sido un aprendizaje bastante satisfactorio
por la utilidad que tienen de cara a mi futuro en el mundo laboral.
Para finalizar las conclusiones me parece oportuno comentar que actualmente
DeliverIt es un prototipo totalmente usable. Está siendo probado en varias asig-
naturas con varias prácticas cada una y con una cantidad importante de alum-
nos, por lo que la tarea principal para la que se desarrolló ya está implementada
y por esto tanto mi trabajo, como los trabajos futuros irán siendo cada vez más
específicos hasta el momento que solo quede el mantenimiento del sistema.