\chapter{Resumen del trabajo realizado}

Inicialmente el plan de trabajo tenía como objetivo resolver las siguientes tareas, dedicando las horas indicadas:

\begin{table}[h]
\centering
\begin{tabular}{|p{1cm}|p{11.9cm}|p{1.1cm}|}
\hline
\textbf{Num} & \textbf{Tarea} & \textbf{Horas}\\
\hline\hline
1 & Formación en tecnologías de visualización y lenguajes de programación necesarios. & 90\\
\hline
2 & Selección e implementación del algoritmo a visualizar. & 30\\
\hline
3 & Diseño de la interacción del cuadro de mandos con el algoritmo a visualizar. & 10\\
\hline
4 & Implementación y pruebas del cuadro de mandos. & 120\\
\hline
5 & Elaboración de la memoria. & 45\\
\hline
6 & Elaboración de la presentación. & 10\\
\hline
7 & Preparación de la defensa del trabajo. & 19\\
\hline\hline
\textbf{} & \textbf{Total} & \textbf{324}\\
\hline
\end{tabular}
\caption{Lista de tareas}
\end{table}

La distribución planeada por semanas es la siguiente:


\begin{figure}[h]
	{\fontsize{3}{4}\selectfont
		\centering      
    	\def\svgscale{0.185}
    	\input{dibujo.pdf_tex}
    	\caption{Diagrama de Gantt}
	}
\end{figure}

A fecha de escritura (semana del 18 de abril) el diagrama de Gantt correspondiente a las tareas realizadas es el siguiente:

\begin{figure}[h]
	{\fontsize{3}{4}\selectfont
		\centering      
    	\def\svgscale{0.185}
    	\input{dibujo.pdf_tex}
    	\caption{Diagrama de Gantt}
	}
\end{figure}

Como se puede observar, las tareas cuya fecha de finalización es posterior a la fecha del informe están completadas, sin embago el inicio de la tarea número 4 (Implementación y pruebas del cuadro de mandos) está retrasada una semana.