\chapter{Resumen del trabajo realizado}
\label{ch:trabajo_realizado}

Inicialmente el plan de trabajo tenía como objetivo resolver las tareas de la tabla~\ref{lista_tareas}, dedicando las horas indicadas.

\begin{table}[h]
\centering
\begin{tabular}{|p{1cm}|p{11cm}|p{1.1cm}|}
\hline
\textbf{Num} & \textbf{Tarea} & \textbf{Horas}\\
\hline\hline
1 & Formación en tecnologías de visualización y lenguajes de programación necesarios. & 90\\
\hline
2 & Selección e implementación del algoritmo a visualizar. & 30\\
\hline
3 & Diseño de la interacción del cuadro de mandos con el algoritmo a visualizar. & 10\\
\hline
4 & Implementación y pruebas del cuadro de mandos. & 120\\
\hline
5 & Elaboración de la memoria. & 45\\
\hline
6 & Elaboración de la presentación. & 10\\
\hline
7 & Preparación de la defensa del trabajo. & 19\\
\hline\hline
\textbf{} & \textbf{Total} & \textbf{324}\\
\hline
\end{tabular}
\caption{Lista de tareas}
\label{lista_tareas}
\end{table}

La planificación inicial contaba con la distribución de tareas por semanas que se muestra en el diagrama de Gantt de la Figura~\ref{plan_trabajo}. Cabe resaltar que la dedicación por semanas de cada tarea no es representativa de las horas semanales dedicadas a dicha tarea. El diagrama es una imagen con formato \texttt{svg}, por tanto se puede aumentar para leer con claridad el contenido.

\begin{figure}[h]
	{\fontsize{3}{4}\selectfont
		\centering
    	\def\svgscale{0.185}
    	\input{imagenes/plan_trabajo.pdf_tex}
    	\caption{Diagrama de Gantt planificado}
    	\label{plan_trabajo}
	}
\end{figure}

A fecha de escritura (semana del 18 de abril) el diagrama de Gantt correspondiente a las tareas realizadas hasta la fecha es el de la Figura~\ref{trabajo_realizado}.

\begin{figure}[H]
	{\fontsize{3}{4}\selectfont
		\centering
    	\def\svgscale{0.185}
    	\chapter{Resumen del trabajo realizado}
\label{ch:trabajo_realizado}

Inicialmente el plan de trabajo tenía como objetivo resolver las tareas de la tabla~\ref{lista_tareas}, dedicando las horas indicadas.

\begin{table}[h]
\centering
\begin{tabular}{|p{1cm}|p{11cm}|p{1.1cm}|}
\hline
\textbf{Num} & \textbf{Tarea} & \textbf{Horas}\\
\hline\hline
1 & Formación en tecnologías de visualización y lenguajes de programación necesarios. & 90\\
\hline
2 & Selección e implementación del algoritmo a visualizar. & 30\\
\hline
3 & Diseño de la interacción del cuadro de mandos con el algoritmo a visualizar. & 10\\
\hline
4 & Implementación y pruebas del cuadro de mandos. & 120\\
\hline
5 & Elaboración de la memoria. & 45\\
\hline
6 & Elaboración de la presentación. & 10\\
\hline
7 & Preparación de la defensa del trabajo. & 19\\
\hline\hline
\textbf{} & \textbf{Total} & \textbf{324}\\
\hline
\end{tabular}
\caption{Lista de tareas}
\label{lista_tareas}
\end{table}

La planificación inicial contaba con la distribución de tareas por semanas que se muestra en el diagrama de Gantt de la Figura~\ref{plan_trabajo}. Cabe resaltar que la dedicación por semanas de cada tarea no es representativa de las horas semanales dedicadas a dicha tarea. El diagrama es una imagen con formato \texttt{svg}, por tanto se puede aumentar para leer con claridad el contenido.

\begin{figure}[h]
	{\fontsize{3}{4}\selectfont
		\centering
    	\def\svgscale{0.185}
    	\input{imagenes/plan_trabajo.pdf_tex}
    	\caption{Diagrama de Gantt planificado}
    	\label{plan_trabajo}
	}
\end{figure}

\newpage

A fecha de escritura (semana del 18 de abril) el diagrama de Gantt correspondiente a las tareas realizadas hasta la fecha es el de la Figura~\ref{trabajo_realizado}.

\begin{figure}[H]
	{\fontsize{3}{4}\selectfont
		\centering
    	\def\svgscale{0.185}
    	\chapter{Resumen del trabajo realizado}
\label{ch:trabajo_realizado}

Inicialmente el plan de trabajo tenía como objetivo resolver las tareas de la tabla~\ref{lista_tareas}, dedicando las horas indicadas.

\begin{table}[h]
\centering
\begin{tabular}{|p{1cm}|p{11cm}|p{1.1cm}|}
\hline
\textbf{Num} & \textbf{Tarea} & \textbf{Horas}\\
\hline\hline
1 & Formación en tecnologías de visualización y lenguajes de programación necesarios. & 90\\
\hline
2 & Selección e implementación del algoritmo a visualizar. & 30\\
\hline
3 & Diseño de la interacción del cuadro de mandos con el algoritmo a visualizar. & 10\\
\hline
4 & Implementación y pruebas del cuadro de mandos. & 120\\
\hline
5 & Elaboración de la memoria. & 45\\
\hline
6 & Elaboración de la presentación. & 10\\
\hline
7 & Preparación de la defensa del trabajo. & 19\\
\hline\hline
\textbf{} & \textbf{Total} & \textbf{324}\\
\hline
\end{tabular}
\caption{Lista de tareas}
\label{lista_tareas}
\end{table}

La planificación inicial contaba con la distribución de tareas por semanas que se muestra en el diagrama de Gantt de la Figura~\ref{plan_trabajo}. Cabe resaltar que la dedicación por semanas de cada tarea no es representativa de las horas semanales dedicadas a dicha tarea. El diagrama es una imagen con formato \texttt{svg}, por tanto se puede aumentar para leer con claridad el contenido.

\begin{figure}[h]
	{\fontsize{3}{4}\selectfont
		\centering
    	\def\svgscale{0.185}
    	\input{imagenes/plan_trabajo.pdf_tex}
    	\caption{Diagrama de Gantt planificado}
    	\label{plan_trabajo}
	}
\end{figure}

\newpage

A fecha de escritura (semana del 18 de abril) el diagrama de Gantt correspondiente a las tareas realizadas hasta la fecha es el de la Figura~\ref{trabajo_realizado}.

\begin{figure}[H]
	{\fontsize{3}{4}\selectfont
		\centering
    	\def\svgscale{0.185}
    	\chapter{Resumen del trabajo realizado}
\label{ch:trabajo_realizado}

Inicialmente el plan de trabajo tenía como objetivo resolver las tareas de la tabla~\ref{lista_tareas}, dedicando las horas indicadas.

\begin{table}[h]
\centering
\begin{tabular}{|p{1cm}|p{11cm}|p{1.1cm}|}
\hline
\textbf{Num} & \textbf{Tarea} & \textbf{Horas}\\
\hline\hline
1 & Formación en tecnologías de visualización y lenguajes de programación necesarios. & 90\\
\hline
2 & Selección e implementación del algoritmo a visualizar. & 30\\
\hline
3 & Diseño de la interacción del cuadro de mandos con el algoritmo a visualizar. & 10\\
\hline
4 & Implementación y pruebas del cuadro de mandos. & 120\\
\hline
5 & Elaboración de la memoria. & 45\\
\hline
6 & Elaboración de la presentación. & 10\\
\hline
7 & Preparación de la defensa del trabajo. & 19\\
\hline\hline
\textbf{} & \textbf{Total} & \textbf{324}\\
\hline
\end{tabular}
\caption{Lista de tareas}
\label{lista_tareas}
\end{table}

La planificación inicial contaba con la distribución de tareas por semanas que se muestra en el diagrama de Gantt de la Figura~\ref{plan_trabajo}. Cabe resaltar que la dedicación por semanas de cada tarea no es representativa de las horas semanales dedicadas a dicha tarea. El diagrama es una imagen con formato \texttt{svg}, por tanto se puede aumentar para leer con claridad el contenido.

\begin{figure}[h]
	{\fontsize{3}{4}\selectfont
		\centering
    	\def\svgscale{0.185}
    	\input{imagenes/plan_trabajo.pdf_tex}
    	\caption{Diagrama de Gantt planificado}
    	\label{plan_trabajo}
	}
\end{figure}

\newpage

A fecha de escritura (semana del 18 de abril) el diagrama de Gantt correspondiente a las tareas realizadas hasta la fecha es el de la Figura~\ref{trabajo_realizado}.

\begin{figure}[H]
	{\fontsize{3}{4}\selectfont
		\centering
    	\def\svgscale{0.185}
    	\input{imagenes/trabajo_realizado.pdf_tex}
    	\caption{Diagrama de Gantt con tareas completadas}
    	\label{trabajo_realizado}
	}
\end{figure}

En este diagrama se pueden observar las semanas en las que se ha trabajado (color verde) o no (color naranja) en determinadas tareas. Como se puede observar, tanto la primera tarea como la tercera están completadas. En cuanto a la cuarta tarea, no se ha trabajado durante las dos primeras semanas puesto que, entre otros factores, se ha tenido que invertir ese tiempo en formación adicional en contenidos relativos a la misma.

    	\caption{Diagrama de Gantt con tareas completadas}
    	\label{trabajo_realizado}
	}
\end{figure}

En este diagrama se pueden observar las semanas en las que se ha trabajado (color verde) o no (color naranja) en determinadas tareas. Como se puede observar, tanto la primera tarea como la tercera están completadas. En cuanto a la cuarta tarea, no se ha trabajado durante las dos primeras semanas puesto que, entre otros factores, se ha tenido que invertir ese tiempo en formación adicional en contenidos relativos a la misma.

    	\caption{Diagrama de Gantt con tareas completadas}
    	\label{trabajo_realizado}
	}
\end{figure}

En este diagrama se pueden observar las semanas en las que se ha trabajado (color verde) o no (color naranja) en determinadas tareas. Como se puede observar, tanto la primera tarea como la tercera están completadas. En cuanto a la cuarta tarea, no se ha trabajado durante las dos primeras semanas puesto que, entre otros factores, se ha tenido que invertir ese tiempo en formación adicional en contenidos relativos a la misma.

    	\caption{Diagrama de Gantt con tareas completadas}
    	\label{trabajo_realizado}
	}
\end{figure}

En este diagrama se pueden observar las semanas en las que se ha trabajado (color verde) o no (color naranja) en determinadas tareas. Como se puede observar, tanto la primera tarea como la tercera están completadas. En cuanto a la cuarta tarea, no se ha trabajado en las dos primeras semanas puesto que, entre otros factores, se ha tenido que invertir ese tiempo en formación adicional en contenidos relativos a la misma.
