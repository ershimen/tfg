\chapter*{Explicación y justificación de las modificaciones al Plan de Trabajo}

Como se puede observar en la figura 1. Esto se debe a que el proceso de formación está llevando bastante más tiempo del esperado. Como se explica en secciones posteriores, se programan los módulos de la aplicación en varios lenguajes de programación y \textit{frameworks} desconocidos. Dado que este proyecto es de los primeros que he implementado con estos lenguajes, lleva más tiempo del habitual programar las funcionalidades requeridas puesto que la experiencia es lo que ayuda a reflejar en código las ideas.

Además de este aspecto, el diseño de la interacción del cuadro de mandos con el algoritmo (la parte más difícil de diseñar) ha sufrido numerosas modificaciones que afectan, con mayor o menor medida, a las demás partes de la aplicación.

\section{Revisión de la lista de objetivos del trabajo}

Los objetivos del trabajo se mantienen (ver si se incluye el de herramienta de depuracion de sistemas distribuidos).

\section{Revisión de la lista de tareas}

La lista de tareas se mantiene, sin embargo se modifican ligeramente las horas:

\section{Revisión del Diagrama de Gantt}

Ante los retrasos actuales en las tareas, el diagrama de Gantt revisado es el siguiente:
