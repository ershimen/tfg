\chapter*{Explicación y justificación de las modificaciones del Plan de Trabajo}

El plan de trabajo se mantiene. Sin embargo, el retraso en la cuarta tarea implica que hasta la fecha prevista de finalización de la misma se tendrá que priorizar el progreso en esta. El retraso se debe, en parte, a la gran cantidad de tiempo que se tiene que dedicar a la formación en las tecnologías necesarias. En la implementación se emplean diversos lenguajes de programación y \textit{frameworks} de los cuales, la mayoría, eran desconocidos. Además de este aspecto, el diseño de la interacción del cuadro de mandos con el algoritmo (la parte más difícil de diseñar) ha sufrido numerosas modificaciones que afectan, con mayor o menor medida, a las demás partes de la aplicación.

\section{Revisión de la lista de objetivos del trabajo}

Los objetivos del trabajo se mantienen. A modo de recordatorio, el objetivo principal es proporcionar una herramienta que facilite la comprensión del funcionamiento de los algoritmos de sistemas distribuidos. Esta herramienta debe permitir al usuario interactuar con el algoritmo en tiempo real para poder observar cuál es la reacción ante distintas situaciones. El proyecto, además de tener un objetivo pedagógico, también puede ser empleado con fines de depuracion por los desarrolladores de estos algoritmos.

\section{Revisión de la lista de tareas}

Tanto la lista de tareas como las horas de dedicación a las mismas se mantiene. Se puede consultar en el cuadro del \hyperref[ch:trabajo_realizado]{capítulo anterior}.

\newpage

\section{Revisión del Diagrama de Gantt}

La distribución de las tareas por semanas se mantiene, puede consultarse en el diagrama de la Figura~\ref{revision_gantt}.

\begin{figure}[H]
	{\fontsize{3}{4}\selectfont
		\centering
    	\def\svgscale{0.26}
    	\input{imagenes/revision_gantt.pdf_tex}
    	\caption{Diagrama de Gantt revisado}
    	\label{revision_gantt}
	}
\end{figure}
